% ---------------------------------------------------------------------- %

\documentclass[letterpaper,10pt]{article}



\pagestyle{empty}

\usepackage[table]{xcolor}
\usepackage{color, colortbl}
\usepackage{tabularx}
\usepackage{amssymb}
\usepackage{enumerate}

\definecolor{LightGray}{gray}{0.9}

\usepackage{amsmath}
\usepackage{amscd}
\usepackage{url}

\usepackage{graphicx}


\title{Observed and predicted North American teleseismic delay times}
\author{Northwestern Seismology Group}
\date{\today}

\begin{document}
\maketitle

% ************************************************************* %
%                                                               %
%                            ABSTRACT                           %
%                                                               %
% ************************************************************* %

\section{About}

\begin{itemize}
  \item AIMBAT used to measure absolute delay times of teleseismic $P$ and $S$ waves 
  \item Recorded by seismic station from EarthScope's USArray, previous PASSCAL arrays, other networks in North America
  \item Estimate contributions to delays from
        \begin{enumerate}
          \item Delays from crustal structure

                Using crustal models
          \item Event side heterogeniety

                Using delay time distribution
        \end{enumerate}
  \item Subtract contributions to delays, from actual measurements $\Rightarrow$ map average delay at each station 
  \item Analyze average delay times to investigate structure of North American mantle, formation of North American continent. 
        \begin{enumerate}
          \item Mantle $S$ delay times from stations west of the Rocky Mountains are 4.2 s larger than delay times from stations within the US portion of stable North America.
          \item $S$ delays at Yellowstone are another 4 s larger than those west of Rocky Mountains. 
          \item Delay time gradients of various steepness coincide with surface geological boundaries. 
        \end{enumerate}
  \item Predictions
        \begin{itemize}
          \item Predictions of teleseismic $S$ delays from 12 3D tomographic mantle $S$ velocity models agree with observed delay time patterns.

          Underestimate delays and advances to varying degrees. 
          \item Similar predictions from tomographic models derived from data other than teleseismic arrival times 

          These overestimate the smoothness of delay time patterns. 
        \end{itemize}
  \item Using tomographic models to predict and study the size and distribution of delay contributions from modeled heterogeneity in different depth ranges. 
        \begin{itemize}
          \item 80 - 240 km depth range is dominant contributor to delay time contrasts and variance

                Corresponds to asthenosphere in western US and lithosphere in central and eastern US. 
          \item Average depth to which observed delays require central US lithosphere to extend is likely shallower than 240 km.

                Consistent with seismic bottom of lithosphere imaged in seismic-velocity models not derived from teleseismic delay times. 
        \end{itemize}
\end{itemize}

% ************************************************************* %
%                                                               %
%                            ABSTRACT                           %
%                                                               %
% ************************************************************* %

% ************************************************************* %
%                                                               %
%                         INTRODUCTION                          %
%                                                               %
% ************************************************************* %

\section{Introduction}

Common features of models 

% ************************************************************* %
%                                                               %
%                         INTRODUCTION                          %
%                                                               %
% ************************************************************* %



















% ------------------------------------------------------------------------- %

\end{document}

% --------------------------------- END --------------------------------- %
